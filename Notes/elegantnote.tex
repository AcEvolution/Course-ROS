\documentclass[geye,green,kindle,cn]{elegantnote}

\usepackage{makecell}
\usepackage{tabularx}

\title{「ROS」学习笔记}
\author{\href{https://accelerator-blog.com}{陈策}}
\version{0.0}
\date{2019 年 11 月 21 日}
\begin{document}
\maketitle
\centerline{\includegraphics[width=0.32\textwidth]{src/TsinghuaLogo.pdf}}

\begin{table}
    \centering
    \caption{中文术语表A} \label{table:ChineseTermTableA}
    \begin{tabular}{p{80pt}<{\centering}p{80pt}<{\centering}}
    \Xhline{1.0pt}
    \textbf{英文术语} & 
    \textbf{中文术语} \\
    \hline
    Node &
    节点 \\
    Master &
    主节点 \\
    Package &
    功能包 \\
    Metapackage &
    元功能包 \\
    Dependent Package &
    依赖包 \\
    Message &
    消息 \\
    Service &
    服务 \\
    Topic &
    话题 \\
    Service Server &
    服务服务器 \\
    Service Client &
    服务客户端 \\
    Parameter Server &
    参数服务器 \\
    Publisher &
    发布者 \\
    Subscriber &
    订阅者 \\
    \Xhline{1.0pt}
    \end{tabular}
\end{table}

\begin{table}
    \centering
    \caption{中文术语表B} \label{table:ChineseTermTableB}
    \begin{tabular}{p{80pt}<{\centering}p{80pt}<{\centering}}
    \Xhline{1.0pt}
    \textbf{英文术语} & 
    \textbf{中文术语} \\
    \hline
    Launch &
    启动 \\
    Client Library &
    客户端库 \\
    Repositories &
    存储库 \\
    Namespace &
    命名空间 \\
    Base Name &
    基本名称 \\
    Global Name &
    全局名称 \\
    Private Name &
    私有名称 \\
    Relative Name &
    相对名称 \\
    Node Handle &
    节点句柄 \\
    Timer &
    计时器 \\
    Transform &
    变换 \\
    Master PC &
    总机 \\
    Host PC &
    主机 \\
    Build &
    构建 \\
    \Xhline{1.0pt}
    \end{tabular}
\end{table}

\begin{table}
    \centering
    \caption{中文术语表C} \label{table:ChineseTermTableC}
    \begin{tabular}{p{80pt}<{\centering}p{80pt}<{\centering}}
    \Xhline{1.0pt}
    \textbf{英文术语} & 
    \textbf{中文术语} \\
    \hline
    Motion Planning &
    运动规划 \\
    Odometry &
    侧位 \\
    Pose &
    姿态 \\
    Play/Replay &
    回放 \\
    Introspection &
    自检 \\
    Icon &
    图标 \\
    Dead Reckoning &
    导航推测 \\
    Base &
    基座 \\
    Link &
    连杆 \\
    Joint &
    关节 \\
    End Effector &
    末端执行器 \\
    \Xhline{1.0pt}
    \end{tabular}
\end{table}

\section{机器人软件平台}
\subsection{平台的组件}
\subsection{机器人软件平台}
\subsection{机器人软件平台的必要性}
\subsection{机器人软件平台将带来的未来}
\section{机器人操作系统 ROS}
\subsection{ROS 简介}
\subsection{元操作系统}
\subsection{ROS的目的}
\subsection{ROS的组件}
\subsection{ROS的生态系统}
\subsection{ROS的历史}
\subsection{ROS的版本}
\subsubsection{版本规则}
\subsubsection{版本周期}
\subsubsection{选择版本}
\section{搭建ROS开发环境}
\subsection{安装ROS}
\subsubsection{常规安装}
\subsubsection{简易安装}
\subsection{搭建ROS开发环境}
\subsubsection{ROS配置}
\subsubsection{集成开发环境(IDE)}
\subsection{ROS操作测试}
\section{ROS的重要概念}
\subsection{ROS术语}
\subsection{消息通信}
\subsubsection{话题(topic)}
\subsubsection{服务(service)}
\subsubsection{动作(action)}
\subsubsection{参数(parameter)}
\subsubsection{消息通信的过程}
\subsection{消息}
\subsubsection{Msg文件}
\subsubsection{Srv文件}
\subsubsection{Action文件}
\subsection{名称(name)}
\subsection{坐标变换(TF)}
\subsection{客户端车}
\subsection{异构设备间的通信}
\subsection{文件系统}
\subsubsection{文件组织结构}
\subsubsection{安装目录}
\subsubsection{工作目录}
\subsection{构建系统}
\subsubsection{创建功能包}
\subsubsection{修改功能包配置文件(package.xml)}
\subsubsection{修改构建配置文件(CMakelists.txt)}
\subsubsection{编写源代码}
\subsubsection{构建功能包}
\subsubsection{运行节点}
\section{ROS命令}
\subsection{ROS命含概述}
\subsection{ROS shell命令}
\subsubsection{Roscd:移动ROS目录}
\subsubsection{Rosls:ROS文件列表}
\subsubsection{Rosed:ROS编辑命令}
\subsection{ROS执行命令}
\subsubsection{Roscore:运行roscore}
\subsubsection{Rosrun:运行ROS节点}
\subsubsection{Roslaunch:运行多个ROS节点}
\subsubsection{Roscrea:检查及删除ROS日志}
\subsection{ROS信息命令}
\subsubsection{运行节点}
\subsubsection{Rosnode:ROS节点}
\subsubsection{Rostoplc:ROS话题}
\subsubsection{Rosservice:ROS服务}
\subsubsection{Rosparam:ROS参数}
\subsubsection{Rosmsg:ROS消息信息}
\subsubsection{Rossrv:ROS服务信息}
\subsubsection{Rosbag:ROS日志信息}
\subsection{ROS catkin命令}
\subsection{ROS功能包命令}
\section{ROS工具}
\subsection{三维可视化工具(RViz)}
\subsubsection{RViz安装与运行}
\subsubsection{RViz画面布局}
\subsubsection{RViz显示屏}
\subsection{ROSGUI开发工具(rqt)}
\subsubsection{Rqt安装与运行}
\subsubsection{Rqt插件}
\subsubsection{Rqt\_image\_view}
\subsubsection{Rqt\_graph}
\subsubsection{Rat\_plot}
\subsubsection{Rqt\_bag}
\section{ROS编程基础}
\subsection{ROS编程前须知事项}
\subsubsection{标准单位}
\subsubsection{坐标表现方式}
\subsubsection{编程规则}
\subsection{发布者节点和订阅者节点的创建和运行}
\subsubsection{创建功能包}
\subsubsection{修改功能包配置文件}
\subsubsection{修改构建配置文件(Cmakelists.txt)}
\subsubsection{创建消息文件}
\subsubsection{创建发布者节点}
\subsubsection{创建订说者节点}
\subsubsection{构建(build)节点}
\subsubsection{运行发布者}
\subsubsection{运行订阅者}
\subsubsection{检查运行中的节点的通信状态}
\subsection{创建和运行服务服务器与客户端节点}
\subsubsection{创建功能包}
\subsubsection{修改功能包配置文件(package.xml)}
\subsubsection{修改构建配置文件(CMakelists.txt)}
\subsubsection{创建服务文件}
\subsubsection{创建服务服务器节点}
\subsubsection{创建服务客户端节点}
\subsubsection{构建节点}
\subsubsection{运行服务服务器}
\subsubsection{运行服务客户端}
\subsubsection{Rosservice call1命令的用法}
\subsubsection{GUI工具ServiceCaller的用法}
\subsection{创建和运行动作服务器和客户端节点}
\subsubsection{生成功能包}
\subsubsection{修改功能包配置文件(package.xml)}
\subsubsection{修改构建配置文件(CMakelists.txt)}
\subsubsection{创建动作文件}
\subsubsection{创建动作服务节点}
\subsubsection{创建客户端节点}
\subsubsection{构建节点}
\subsubsection{运行动作服务器}
\subsubsection{运行动作客户端}
\subsection{参数的用法}
\subsubsection{利用参数创建节点}
\subsubsection{设置参数}
\subsubsection{读取参数}
\subsubsection{构建节点和运行节点}
\subsubsection{查看参数目录}
\subsubsection{参数的用例}
\subsection{Roslaunch的用法}
\subsubsection{Roslaunch的应用}
\subsubsection{Launch标签}
\section{机器人、传感器和电机}
\subsection{机器人功能包}
\subsection{传感器功能包}
\subsubsection{传感器的类型}
\subsubsection{传感器功能包的分类}
\subsection{相机}
\subsubsection{USB摄像头相关功能包}
\subsubsection{USB摄像头测试}
\subsubsection{查看图像信息}
\subsubsection{远程传输图像}
\subsubsection{相机校准}
\subsection{深度相机(Depth Camera)}
\subsubsection{Depth Camera的类型}
\subsubsection{Depth Camera测试}
\subsubsection{Point Cloud Data(点云数据)的可视化}
\subsubsection{Point Cloud Data相关库}
\subsection{激光距离传感器}
\subsubsection{LDS传感器距离测量原理}
\subsubsection{LDS测试}
\subsubsection{可视化LDS的距离值}
\subsubsection{LDS的应用}
\subsection{电机功能包}
\subsubsection{Dynamixel舵机}
\subsection{已公开的功能包的用法}
\subsubsection{搜索功能包}
\subsubsection{安装依赖包}
\subsubsection{安装功能包}
\subsubsection{运行功能包}
\section{嵌入式系统}
\subsection{OpenCR}
\subsubsection{特点}
\subsubsection{控制板规格}
\subsubsection{搭建开发环境}
\subsubsection{OpenCR例程}
\subsection{RosseriaL}
\subsubsection{Rosserial server}
\subsubsection{Rosserial client}
\subsubsection{Rosserial协议}
\subsubsection{Rosserial的约束条件}
\subsubsection{安装rosserial}
\subsubsection{Rosserial例程}
\subsection{TurtleBot3的固件}
\subsubsection{TurtleBot3 Burgeri固件}
\subsubsection{TurtleBot3 Waffle和Waffle Pi固件}
\subsubsection{Turtlebot3配置固件}
\section{移动机器人}
\subsection{ROS支持的机器人}
\subsection{TurtleBot3系列机器人}
\subsection{TurleBot3的硬件}
\subsection{Turtlebot3软件}
\subsection{Turtlebot3的开发环境}
\subsection{Turtlebot3远程控制}
\subsubsection{遥控TurtleBot3}
\subsubsection{可视化TurtleBot3}
\subsection{Turtlebot3话题}
\subsubsection{订阅话题}
\subsubsection{通过订阅话题控制机器人}
\subsubsection{发布话题}
\subsubsection{通过发布话题识别机器人状态}
\subsection{使用RViz仿真Turtlebot3}
\subsubsection{仿真}
\subsubsection{运行虚拟机器人}
\subsubsection{Odometry和TF}
\subsection{利用Gazebo仿真Turtlebot3}
\subsubsection{Gazebo仿真器}
\subsubsection{启动虚拟机器人}
\subsubsection{虚拟SLAM和导航}
\section{SLAM和导航}
\subsection{导航及其组成要素}
\subsubsection{移动机器人的导航}
\subsubsection{地图}
\subsubsection{测量或估计机器人姿态的功能}
\subsubsection{识别障碍物如墙壁和物体}
\subsubsection{计算最优路径和行驶功能}
\subsection{SLAM实习篇}
\subsubsection{对于使用SLAM的机器人的硬件限制}
\subsubsection{SLAM的实验环境}
\subsubsection{用于SLAM的ROS功能包}
\subsubsection{运行SLAM}
\subsection{利用预先准备好的bag文件运行的SLAM}
\subsection{SLAM应用篇}
\subsubsection{地图}
\subsubsection{SLAM所需的信息}
\subsubsection{SLAM的处理过程}
\subsubsection{坐标变换(TF)}
\subsubsection{Turtlebot3\_slam功能包}
\subsection{SLAM理论篇}
\subsubsection{SLAM}
\subsubsection{多种位置估计(localization)方法论}
\subsection{导航实战篇}
\subsubsection{用于导航的ROS功能包}
\subsubsection{运行导航}
\subsection{导航应用程序}
\subsubsection{导航}
\subsubsection{导航所需的信息}
\subsubsection{Turtlebot3\_navigation的各节点和话题状态}
\subsubsection{Turtlebot3\_navigation设置}
\subsubsection{设置turtlebot3\_navigation的详细参数}
\subsection{导航理论篇}
\subsubsection{Costmap}
\subsubsection{AMCL}
\subsection{Dynamic Window Approach(DWA)}
\section{服务机器人}
\subsection{配关服务机器人}
\subsection{关服务机器人的结构}
\subsubsection{系统结构}
\subsubsection{系统设计}
\subsubsection{服务核心节点}
\subsubsection{服务主节点}
\subsubsection{服务从节点}
\subsection{用ROS Java进行Android平板PC编程}
\section{机械手臂}
\subsection{机械手臂介绍}
\subsubsection{机械手臂的结构和控制}
\subsubsection{机械手臂和ROS}
\subsection{OpenManipulator。建模和仿真}
\subsubsection{OpenManipulator}
\subsubsection{机械手臂建模}
\subsubsection{Gazebo设置}
\subsection{MoveIt!}
\subsubsection{Move\_group}
\subsubsection{MoveIt!Setup Assistant}
\subsubsection{Gazebo仿真}
\subsection{应用于实际平台}
\subsubsection{准备和控制OpenManipulator}
\subsubsection{OpenManipulator与turtleBot3Waffle及Waffle Pi}
\end{document}